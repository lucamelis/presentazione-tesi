\documentclass{beamer}

%%% Dichiarazione dei pacchetti standard.
\usepackage[italian]{babel}

%%% Personalizzazione del layout---articolata su cinque livelli.
\usetheme{split}        % layout complessivo. 
\useinnertheme{default} % layout interno.
\useoutertheme{default} % layout esterno.
\usecolortheme{default} % schema di colori.
\usefonttheme{default}  % schema dei font.
% Inutile dire che se volete tutti i default, potete risparmiarvi gli ultimi
% quattro comandi. 

%%% Titolo e autore.
\title{Creare presentazioni in \LaTeX}
\subtitle{La classe {\tt beamer}}
\author{Luca Baldini}
\institute{Gruppo Utilizzatoti Italiani di \TeX}
\date{\today}



\begin{document}

\begin{frame}
  \titlepage
\end{frame}


\section[Sommario]{}
\begin{frame}
  \tableofcontents
\end{frame}


\section{Introduzione}
\begin{frame}
  \frametitle{Introduzione}
  \framesubtitle{Il nome del gioco}
  {\tt beamer} \`e un pacchetto estremamente potente e flessibile per
  la realizzazzione di presentazioni. Un paio di caratteristiche degne di nota:
  \begin{itemize}
  \item D\`a accesso a tutte le funzionalit\`a di base di \LaTeX e pu\`o
    essere utilizzato insieme a (quasi) tutti gli altri pacchetti.
    \begin{itemize}
    \item Supporta il sezionamento.
    \item Molti dei costrutti pi\`u usuali sono ridefiniti in modo da
      supportare l'\emph{overlay}.
    \end{itemize}
  \item Fornisce strumenti aggiuntivi mirati alle presentazioni (barra di
    navigazione, dissolvenze, controllo dell'opacit\`a).
    \begin{itemize}
    \item Se il tema di \emph{default} vi sembra troppo pesante, non
      preoccupatevi: avete ampie possibilit\`a di personalizzazione!
    \end{itemize}
\end{itemize}
\end{frame}


\section{Una breve panoramica}
\subsection{Funzionalit\`a di base}
\begin{frame}
  \frametitle{Funzionalit\`a di base}
  \framesubtitle{Elenchi e formule}
  Si possono inserire banalmente elenchi numerati\ldots
  \begin{enumerate}
  \item Punto primo.
  \item Punto secondo.
  \item Punto terzo.
  \end{enumerate}

  \ldots e formule\footnote{A proposito\ldots avete notato l'errore
    in questa?}:
  $$
  \frac{1}{\sqrt{2\pi \sigma}} \int_{-\infty}^{\infty}
  e^{-\frac{(x-\mu)^2}{2\sigma^2}} = 1
  $$
\end{frame}

\subsection{Funzionalit\`a avanzate}
\begin{frame}
  \frametitle{Funzionalit\`a avanzate}
  \framesubtitle{\emph{Overlays}}
  Come promesso:
  \begin{itemize}
  \item<1-> \`E facilissimo\ldots
  \item<2-> \ldots fare\ldots
  \item<3-> \ldots \emph{overlays}\/!      
  \end{itemize}

  \vspace{0.15\textheight}

  \pause[3]
  \setbeamercolor{uppercolor}{fg=white, bg=black}
  \setbeamercolor{lowercolor}{fg=white, bg=blue!50!black}
  \begin{beamerboxesrounded}[upper=uppercolor, lower=lowercolor, shadow=true]
    {E questo \`e niente!}
    date un'occhiata alla documentazione per avere un'idea delle sterminate
    potenzialit\`a che vi vengono offerte.
  \end{beamerboxesrounded}
\end{frame}

\subsection{Personalizzazione}
\begin{frame}[fragile]  % [fragile] \`e necessario per il verbatim!
  \frametitle{Personalizzazzione}
  \framesubtitle{Come diavolo faccio a togliere la barra di navigazione?}
  In fondo non \`e male, ma se proprio volete\ldots avete cinque livelli
  di personalizzazione accessibili tramite i comandi:
  \begin{verbatim}
    \usetheme[opzioni]{nome del tema}
    \useinnertheme[opzioni]{nome del tema}
    \useoutertheme[opzioni]{nome del tema}
    \usecolortheme[opzioni]{nome del tema}
    \usefonttheme[opzioni]{nome del tema}
  \end{verbatim}

  Parlando d'altro\ldots se volete usare l'ambiente {\tt verbatim} dovete
  usare la \emph{keyword} {\tt fragile} all'apertura del \emph{frame}
  (vedi il sorgente).
\end{frame}

\section{Conclusioni}
\begin{frame}
  \begin{itemize}
  \item Da provare assolutamente!!!
  \item E se ancora non vi bastasse\ldots
    \htmladdnormallink{cliccate qui!}{http://latex-beamer.sourceforge.net/}%
    \footnote{Questo presuppone che il vostro \emph{viewer} pdf sia
      configurato opportunamente.}
  \end{itemize}
\end{frame}

\end{document}

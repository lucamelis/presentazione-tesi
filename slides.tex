\section{Introduzione}

\begin{frame}
 \frametitle{Scenario}
 \begin{block}{Cryptography schemes}
 \begin{itemize}
  \item address the security of communication across an insecure medium
  \item are usually based only on complexity assumptions (standard model)
 \end{itemize}
 \end{block}
 
 \begin{block}{Near Future:}
 \begin{itemize}
    \item \alert{Problem:} What if someone constructs large quantum computers?
    \item Cryptography world may fall apart:
    \begin{enumerate}
     \item cryptographic assumptions broken by efficient quantum algorithms
     \item proof of security becomes invalid 
    \end{enumerate}
 \end{itemize}
\end{block}

\end{frame}

\begin{frame}
\frametitle{Post-Quantum cryptography}
 Schemes that are believed to resist classical \& quantum computers
      \begin{itemize}
      \item \textbf{Hash-based cryptography}
      \item \textbf{Code-based cryptography}
      \item \alert{\textbf{Lattice-based cryptography}}
      \end{itemize}
 \begin{block}{\textbf{Our contribution}}
 \begin{itemize}
  \item We investigate about the \alert{Learning Parity with Noise} ($\LPN$) problem
  \item We propose a {\color{blue}{Threshold Public-Key Encryption}} scheme based on $\LPN$
  \item We propose a {\color{blue}{Commitment protocol}} based on $\LPN$
 \end{itemize}

\end{block}

\end{frame}


\section{Learning Parity with Noise Problem $\LPN$}

\begin{frame}
\frametitle{Learning Parity with Noise Problem $\LPN$}
\begin{itemize}
 \item Dimension $\ell$ (security parameter), $q \gg \ell$, $\tau \in \left( 0,\frac{1}{2} \right]$
 \vspace*{5pt}
 \item \textbf{Search}: \underline{find}  $ \vect{s} \in \zol$ given ``noisy random inner products'' \\
 \begin{overprint}
 \onslide<2>
 \begin{align*}
  {\color{blue}{\vect{a_1}}} \R \zol \quad , \quad {\color{red}{b_1}} &= <{\color{blue}{\vect{a_1}}} \: , \: \vect{s}> \: \xor \: e_1 \\  
  \end{align*}

  \onslide<3>
  \begin{align*}
  {\color{blue}{\vect{a_1}}} \R \zol \quad , \quad {\color{red}{b_1}} &= <{\color{blue}{\vect{a_1}}} \: , \: \vect{s}> \: \xor \: e_1 \\
   & \vdots \\
   {\color{blue}{\vect{a_q}}} \R \zol \quad, \quad  {\color{red}{b_q}} &= <{\color{blue}{\vect{a_q}}} \: , \: \vect{s}> \: \xor \: e_q \\
  \end{align*}

  \onslide<4>
  \vspace*{15pt}
\[
{\color{blue} \vect{A}} = \begin{pmatrix}
            & {\color{blue}\vect{a_1}} & \\
            & \vdots  & \\ 
            & {\color{blue}\vect{a_q}} & 
           \end{pmatrix}  , {\color{red} \vect{b}} = {\color{blue} \vect{A}} \cdot \vect{s} \xor \vect{e} 
\]
 \end{overprint}
  Errors $e_i \gets \Ber_{\tau}$, i.e. $\Pr(e_i=1)= \tau$
\end{itemize}
\vspace{5pt}
  \begin{itemize}
    \item \textbf{Decision}: \underline{distinguish} $\left( {\color{blue}\vect{a_i}}, {\color{red}\vect{b_i}} \right)$ from {\color{blue}{uniform}} $( {\color{blue}\vect{a_i}}, $ {\color{blue} $b_i$} $)$ 
\vspace{5pt}
    \item \emph{decisional} and \emph{search} $\LPN$ are \emph{``polinomially equivalent''} \\ 
 \end{itemize}
\end{frame}

\begin{frame}
 \frametitle{Learning Parity with Noise Problem $\LPN$}

  \begin{block}{Hardness of $\LPN$}
  Breaking the search $\LPN$ problem takes
  

% \begin{table}
%     \begin{tabular}{cc}
%     \toprule
%     time & queries ($q$)\\
%     \midrule
%     $2^{\mathcal{\varTheta}(\mathcal{\ell / \log{\ell}})}$ &  $2^{\mathcal{\varTheta}(\mathcal{\ell / \log{\ell}})}$ \\
%     $2^{\mathcal{\varTheta}(\mathcal{\ell / \log{ \log {\ell} } })}$ & $ poly(\ell)$ \\
%     $2^{\mathcal{\varTheta}(\mathcal{\ell})}$ & $ \mathcal{\varTheta}(\ell)$ \\
%     \bottomrule
%     \end{tabular}
% \end{table}


   \begin{itemize}
    \item {\color{blue}{$2^{\mathcal{\varTheta}(\mathcal{\ell / \log{\ell}})}$}} having the same number of samples $q$
    \item {\color{blue}{$2^{\mathcal{\varTheta}(\mathcal{\ell / \log{ \log {\ell} } })}$}} having $q=poly(\ell)$ samples
    \item {\color{blue}{$2^{\mathcal{\varTheta}(\mathcal{\ell})}$}} having $q= \mathcal{\varTheta}(\ell)$ samples
   \end{itemize} 
where $\ell$ is the security parameter
  \end{block}

  \begin{block}{Interesting features}
    \begin{itemize}
      \item \textbf{Efficiency} $\Rightarrow$ suitable for limited computing power devices (e.g. RFID).
      \item \textbf{quantum algorithm resistance}
    \end{itemize}
  \end{block}

\end{frame}

\section{Threshold Public-Key Encryption}

\begin{frame}
\frametitle{Threshold Public-Key Encryption schemes}
\begin{block}{Public-key cryptography}
  \begin{itemize}
   \item the ability of decrypting or signing is restricted to the owner of the secret key.
   \item $\Rightarrow$ \alert{only one person has all the power} 
  \end{itemize}  
 \end{block}
 \end{frame}
 

 \begin{frame}
 \frametitle{Threshold Public-Key Encryption schemes}
 \begin{block}{Solution: Threshold PKE}
 \begin{itemize}
  \item Shares trust among a group of users, such that \emph{enough} of them, the \emph{threshold}, is needed to sign or decrypt
  \item The secret key is split into shares and each share is given to a group of users.
 \end{itemize}  
 \end{block}
 
 \begin{block}{\textbf{Our contribution}}
 A {\color{red}{Threshold Public-Key Encryption}} scheme which is:
 \begin{itemize}
   \item based on $\LPN$
   \item secure in the \emph{Semi-honest} model
  \end{itemize}

 \end{block}

\end{frame}

\begin{frame}
\frametitle{Alekhnovich PKE scheme}

\begin{block}{\textbf{Key Generation}}
 The sender $\mathtt{S}$ chooses
 \begin{itemize}
  \item a secret key $\vect{s} \R \zol$
  \item $\vect{A} \R \Z^{q \times \ell}$ and the error $\vect{e} \gets \Ber_{\tau}^{q}$, where $ \tau \in \Theta(\frac{1}{\sqrt{\ell}}) $ \\
  and computes the \emph{pk} as $\left( \vect{A},\vect{b}= \vect{A}\vect{s} \xor \vect{e} \right)$
 \end{itemize}
\end{block}
%   \begin{figure}  
%     \begin{protocol}{2}
%       \protocolheader{\textbf{Key Generation}}\\
%       \participants{Sender \underline{$\mathtt{S}$}}{Receiver \underline{$\mathtt{R}$}}\\
%       &	\hspace*{3cm} & \mbox{Choose a secret key $\vect{s} \R \zol$ }\\
%       & \quad & \mbox{choose a matrix $\vect{A} \R \Z^{q \times \ell}$}\\
%       & \quad & \mbox{and an error vector $\vect{e} \gets \Ber_{\tau}^{q}$} \\
%     \end{protocol}
%  \end{figure}

  \begin{figure}
  
    \begin{protocol}{2}
      \protocolheader{\textbf{Encryption} of a message bit $m \in \Z$}\\
      \participants{Sender \underline{$\mathtt{S}$}}{Receiver \underline{$\mathtt{R}$}}\\
      \mbox{choose a vector $\vect{f} \gets \Ber_{\tau}^{q}$} & &  \\
      \mbox{compute $\vect{u}= \vect{f} \cdot \vect{A} $ } & & \\
      \mbox{ $c= \left\langle \vect{f},\vect{b} \right\rangle \xor m$} & \sends{( \vect{u}, c ) } \\
    \end{protocol}   
  
 \end{figure}

 \begin{block}{\textbf{Decryption}}
 The receiver $\mathtt{R}$ computes $d = c \xor \left\langle \vect{s},\vect{u} \right\rangle = 
 \dots = \left\langle  \vect{f} , \vect{e} \right\rangle \xor m$ \\ 
\end{block}
\end{frame}

\begin{frame}
\frametitle{Protocol phases}
  \begin{columns}[c]
    \column{0.50\textwidth}
  \begin{itemize}
 \item<1> \alert<1,1>{ \textbf{Key Generation} }
 \item<2> \alert<2,2>{ \textbf{Key Assembly} }
 \item<3-5> \alert<3-5>{ \textbf{Encryption} }
 \item<6-10> \alert<6-10>{ \textbf{Partial Decryption} } 
 \item<11> \alert<11,11>{ \textbf{Finish Decryption} }
 \end{itemize} 
    \column{0.50\textwidth}
\begin{overprint}
 \onslide<1> \begin{center}\pgfuseimage{key_gen}\end{center}
 \onslide<2> \begin{center}\pgfuseimage{key_assembly}\end{center}
 \onslide<3-5> \begin{center}\pgfuseimage{enc}\end{center}
 \onslide<6-10> \begin{center}\pgfuseimage{part_dec}\end{center}
 \onslide<11> \begin{center}\pgfuseimage{finish}\end{center}
\end{overprint}

\end{columns}
\\
%  \begin{columns}[c]
%  \column{0.5\textwidth}
 
   \begin{overprint}
 \onslide<1>
 \begin{block}{Key Generation}
  \begin{itemize}
   \item All the receivers share a matrix $\vect{A}  \R \Z^{q \times \ell}$
   \item Each receiver $\mathtt{R_i}$ \alert{indipendently} choose a secret key $\vect{s_i} \R \zol$ and an error $\vect{e_i} \gets \Ber_{\tau}^{q}$
   \item the public key for $\mathtt{R_i}$ is the pair $\left( \vect{A},\vect{b_i}=\vect{A}\vect{s_i} \xor \vect{e_i} \right)$
  \end{itemize}  
  \end{block}
  
 \onslide<2>
 \begin{block}{Key Assembly}  
  The combined public key is the pair $\left( \vect{A},\vect{b} \right)$, where 
  \[
  \vect{b} = \bigoplus_{i \in I} \vect{b_i} 
   
  \]
and $I$ is the users subset
  \end{block}
  
	    \onslide<3>
	    \begin{figure}
	    
	    \begin{protocol}{2}
% 		\protocolheader{Encryption Phase}\\
		\participants{Sender \underline{$\mathtt{S}$}}{Receivers \underline{$\mathtt{R_i},\mathtt{R_j}$ }} \\
		\left( \vect{C_1},\vect{c_2} \right) \gets \alert{\mathtt{ThLPN.Enc}}(m,\vect{b}) & \phantom{\sends{(\vect{C_1},\vect{c_2})} }\\    
	      \end{protocol} 
	    
	    \end{figure}
	    
	    \onslide<4-5>	    
	    \begin{figure}
	    
	    \begin{protocol}{2}
% 		\protocolheader{Encryption Phase}\\
		\participants{Sender \underline{$\mathtt{S}$}}{Receivers \underline{$\mathtt{R_i},\mathtt{R_j}$ }} \\
		\left( \vect{C_1},\vect{c_2} \right) \gets \alert{\mathtt{ThLPN.Enc}}(m,\vect{b}) & \phantom{\sends{(\vect{C_1},\vect{c_2})} }\\    
	      \end{protocol} 
	    
	    \end{figure}

	    \begin{block}{Encryption function (Alekhnovich scheme)}
	    \[
	    \vect{C_1} &= \vect{F} \cdot \vect{A}, \: \vect{c_2} &= \vect{F} \cdot \vect{b} \xor \underbrace{\begin{bmatrix} 1 \\ \dots \\ 1 \end{bmatrix}}_{q} \cdot m \mbox{ \qquad where $\vect{F} := \begin{bmatrix} \vect{f_1} \\ \dots \\ \vect{f_q} \end{bmatrix}$, $\vect{f_i} \gets \Ber_{\tau}^{q}$}
	    \]
	      \end{block}
	    
	    \onslide<5>
	    \begin{figure}
	    \begin{protocol}{2}
% 		\protocolheader{Encryption Phase}\\
		\participants{Sender \underline{$\mathtt{S}$}}{Receivers \underline{$\mathtt{R_i},\mathtt{R_j}$ }} \\
		\left( \vect{C_1},\vect{c_2} \right) \gets \alert{\mathtt{ThLPN.Enc}}(m,\vect{b}) &  \sends{(\vect{C_1},\vect{c_2})}   \\    
	      \end{protocol}
	    \end{figure}
	    \begin{block}{Encryption function (Alekhnovich scheme)}
	    \[
	    \vect{C_1} &= \vect{F} \cdot \vect{A}, \: \vect{c_2} &= \vect{F} \cdot \vect{b} \xor \underbrace{\begin{bmatrix} 1 \\ \dots \\ 1 \end{bmatrix}}_{q} \cdot m \mbox{ \qquad where $\vect{F} := \begin{bmatrix} \vect{f_1} \\ \dots \\ \vect{f_q} \end{bmatrix}$, $\vect{f_i} \gets \Ber_{\tau}^{q}$}
	    \]
	      \end{block}

\onslide<6>
\begin{figure}
   
    \begin{protocol}{2}
%      \protocolheader{Partial Decryption Phase}\\
      \participants{Receiver \underline{$\mathtt{R_i}$}}{Receiver \underline{$\mathtt{R_j}$ }} \\
      \vect{d_i} \gets \alert{\mathtt{ThLPN.Pdec}}(\vect{C_1},\vect{c_2},\vect{s_i})  & \phantom{\sends{\vect{d_i}}}   \\
      & \phantom{\receives{\vect{d_j}}} & \phantom{\vect{d_j} \gets} \\ 
      & & \phantom{\mathtt{ThLPN.Pdec}(\vect{C_1},\vect{c_2},\vect{s_j})} \\
    \end{protocol}
  
\end{figure}

\onslide<7>
\begin{figure}
   
    \begin{protocol}{2}
%      \protocolheader{Partial Decryption Phase}\\
      \participants{Receiver \underline{$\mathtt{R_i}$}}{Receiver \underline{$\mathtt{R_j}$ }} \\
      \vect{d_i} \gets \alert{\mathtt{ThLPN.Pdec}}(\vect{C_1},\vect{c_2},\vect{s_i})  & \phantom{\sends{\vect{d_i}}}   \\
      & \phantom{\receives{\vect{d_j}}} & \phantom{\vect{d_j} \gets} \\ 
      & & \phantom{\mathtt{ThLPN.Pdec}(\vect{C_1},\vect{c_2},\vect{s_j})} \\
    \end{protocol}
  
\end{figure}
\begin{block}{Partial decryption function (Alekhnovich scheme)}
\[
 \vect{d_i} = \vect{C_1} \cdot \vect{s_i} \xor \vect{\nu_{i}} 
\]
 where  $ \vect{\nu_i} \gets \Ber_{\sigma}^{q} $ 
\end{block}


\onslide<8>
\begin{figure}
  
    \begin{protocol}{2}
%        \protocolheader{Partial Decryption Phase}\\
      \participants{Receiver \underline{$\mathtt{R_i}$}}{Receiver \underline{$\mathtt{R_j}$ }} \\
      \vect{d_i} \gets \alert{\mathtt{ThLPN.Pdec}}(\vect{C_1},\vect{c_2},\vect{s_i})  & \sends{\vect{d_i}}   \\
      & \phantom{\receives{\vect{d_j}}} & \phantom{\vect{d_j} \gets} \\ 
      & & \phantom{\mathtt{ThLPN.Pdec}(\vect{C_1},\vect{c_2},\vect{s_j})} \\
    \end{protocol}
  
\end{figure}
\begin{block}{Partial decryption function (Alekhnovich scheme)}
\[
 \vect{d_i} = \vect{C_1} \cdot \vect{s_i} \xor \vect{\nu_{i}} 
\]
 where  $ \vect{\nu_i} \gets \Ber_{\sigma}^{q} $ 
\end{block}


\onslide<9>
\begin{figure}
  
    \begin{center}
    
    \end{center}
    \begin{protocol}{2}
%        \protocolheader{Partial Decryption Phase}\\
      \participants{Receiver \underline{$\mathtt{R_i}$}}{Receiver \underline{$\mathtt{R_j}$ }} \\
      \vect{d_i} \gets \alert{\mathtt{ThLPN.Pdec}}(\vect{C_1},\vect{c_2},\vect{s_i})  & \sends{\vect{d_i}}   \\
      & \phantom{\receives{\vect{d_j}}} & \vect{d_j} \gets \\ 
      & & \alert{\mathtt{ThLPN.Pdec}}(\vect{C_1},\vect{c_2},\vect{s_j}) \\
    \end{protocol}
  
\end{figure}
\begin{block}{Partial decryption function (Alekhnovich scheme)}
\[
 \vect{d_i} = \vect{C_1} \cdot \vect{s_i} \xor \vect{\nu_{i}} 
\]
 where  $ \vect{\nu_i} \gets \Ber_{\sigma}^{q} $ 
\end{block}



\onslide<10>
\begin{figure}
  
    \begin{protocol}{2}
%       \protocolheader{Partial Decryption Phase}\\
      \participants{Receiver \underline{$\mathtt{R_i}$}}{Receiver \underline{$\mathtt{R_j}$ }} \\
      \vect{d_i} \gets \alert{\mathtt{ThLPN.Pdec}}(\vect{C_1},\vect{c_2},\vect{s_i})  & \sends{\vect{d_i}}   \\
      & \receives{\vect{d_j}} & \vect{d_j} \gets \\ 
      & & \alert{\mathtt{ThLPN.Pdec}}(\vect{C_1},\vect{c_2},\vect{s_j}) \\
    \end{protocol}
  
\end{figure}
\begin{block}{Partial decryption function (Alekhnovich scheme)}
\[
 \vect{d_i} = \vect{C_1} \cdot \vect{s_i} \xor \vect{\nu_{i}} 
\]
 where  $ \vect{\nu_i} \gets \Ber_{\sigma}^{q} $ 
\end{block}


% \end{overprint}
% \begin{overprint}
 


 \onslide<11>
 \begin{block}{Finish decryption}
 \begin{itemize}
  \item Each receiver indipendently computes the vector 
 \begin{eqnarray*}
 \vect{d} =  \vect{c_2} \bigoplus_{i \in I} \left( \vect{d_i} \right)  
 	  =  \vect{F}\cdot \vect{e} \xor \underbrace{\begin{bmatrix} 1 \\ \dots \\ 1 \end{bmatrix}}_{q} \cdot m \bigoplus_{i \in I} \left( \vect{\nu_i} \right) .
 \end{eqnarray*}
 \item the bit in the vector $\vect{d}$ that is in majority is separately chosen by each receiver as the plaintext $m$
 \end{itemize}
 \end{block}
 

 \end{overprint}

  
%  \end{columns}

\end{frame}

% \begin{frame}
% 
% \onslide<2,3>{
% \begin{block}{Encryption function (Alekhnovich scheme)}
%  \begin{align*}
%   \vect{C_1} &= \vect{F} \cdot \vect{A}, \\
%   \vect{c_2} &= \vect{F} \cdot \vect{b} \xor \begin{bmatrix} 1 \\ \dots \\ 1 \end{bmatrix} \cdot m 
%  \end{align*}
% where $\vect{F} := \begin{bmatrix} \vect{f_1} \\ \dots \\ \vect{f_q} \end{bmatrix}$, $\vect{f_i} \gets \Ber_{\tau}^{q}$
%  \end{block}
% }
% 
% \end{frame}
% 
% \begin{frame}
% 
% 
% \only<6>{
% \begin{block}{Finish decryption}
% \begin{itemize}
%  \item Each receiver indipendently computes the vector 
% \begin{eqnarray*}
% \vect{d} =  \vect{c_2} \bigoplus_{i \in I} \left( \vect{d_i} \right)  
% 	  =  \vect{F}\cdot \vect{e} \xor \begin{bmatrix} 1 \\ \dots \\ 1 \end{bmatrix} \cdot m \bigoplus_{i \in I} \left( \vect{\nu_i} \right) .
% \end{eqnarray*}
% \item the bit in the vector $\vect{d}$ that is in majority is separately chosen by each receiver as the plaintext $m$
% \end{itemize}
% \end{block}
% }
% \end{overprint}
% 
% 
% \end{frame}

\begin{frame}
 \frametitle{Protocol Security Analysis }

  \begin{block}{Semi-honest model}
    A semi-honest party:
    \begin{enumerate}
      \item Follows the protocol properly
      \item Keeps a record of all its intermediate computations
    \end{enumerate}
  \end{block}

  \begin{block}{Security}
  \begin{itemize}
    \item \textbf{Encryption:} from the Alekhnovich's scheme security
    \item \textbf{Decryption:} from the $\LPN$ hardness assumption, as each $\mathtt{R}_i$ is generating $\LPN$ samples
    \[
     \vect{d_i} = \vect{C_1} \cdot \vect{s_i} \xor \vect{\nu_{i}} 
    \]

  \end{itemize}
  \end{block}
  
\end{frame}

\begin{frame}
 \frametitle{Protocol Security Analysis }
\begin{block}{Relaxed Semi-honest model}
   \begin{itemize}
    \item Semi-honest model not so realistic (\emph{replay attacks} may occur)
    \item \alert{Problem:} if the same message is encrypted multiple times then it is possible to recover information about the secret key from the ciphertexts\\
% 	\item	e.g. let $\vect{d^{i}_1}$ be $3$ encryptions of the same message $m$:
% 		\begin{align*}
% 		\vect{d_1^1} &= \vect{C_1} \cdot \vect{s_1} \xor \vect{\nu_1^1}, \\
% 		\vect{d_1^2} &= \vect{C_1} \cdot \vect{s_1} \xor \vect{\nu_1^2}, \\
% 		\vect{d_1^3} &= \vect{C_1} \cdot \vect{s_1} \xor \vect{\nu_1^3} 
% 		\end{align*}
% 		where $\vect{\nu_1^i},i=1,\dots,3$ are sampled each time according to $\Ber_\tau^q$.\\
% 	\Rightarrow 
   \end{itemize}
  \end{block}

 \begin{block}{Proposed solutions}
    \begin{enumerate}
     \item implement the receivers as \emph{stateful} machines (not good in resource-constrained devices)
     \item make use of pseudorandom functions (i.e. deterministic algorithms that simulate truly random functions, given a ``seed'')
    \end{enumerate}
 \end{block}
 
\end{frame}

\section{Commitment Protocol}

\begin{frame}
\frametitle{Commitment Protocols}
\begin{block}{Commitment protocol}
 \begin{itemize}
  \item can be thought as the digital analogue of a sealed envelope
  \item \textbf{Commit: } the sender $\mathtt{S}$ commit to a message $m$ and the receiver $\mathtt{R}$ does not learn any information about $m$ (\alert{hiding} property)
  \item \textbf{Open: } $\mathtt{S}$ can choose to open the commitment and reveal the content $m$, but no other value (\alert{binding} property)
 \end{itemize}
\end{block}


\begin{block}{\textbf{Our contribution}}
 We presented a {\color{red}{Commitment protocol}}
 \begin{itemize}
  \item based on the commitment protocol by \textit{Jain et al}
  \item based on Exact-$\LPN$ problem (where $\mathbf{wt}(\vect{e})=\lfloor \tau \cdot \ell \rceil$)
  \item not in a \emph{common reference string (CRS)} model
 \end{itemize}
\end{block}


\end{frame}



\begin{frame}
\frametitle{The commitment protocol by \textit{Jain et al}} 

\begin{block}{Setup Phase}
In order to commit a message $\vect{m} \in \zok$ where $k \in \Theta (\ell + v)$\\
We let $\vect{A'} \R \Z^{k \times \ell}$ and $\vect{A''} \R \Z^{k \times v}$.
We state $\vect{A}= \left[ \vect{A'} \lVert \vect{A''} \right] \in \Z^{k \times (\ell + v)}$ as \alert{the common reference string (CRS)}.
Finally, we set $w = \lfloor\tau k\rceil$.
\end{block}

\begin{figure}
    \begin{protocol}{2}
    \protocolheader{\textbf{Commitment phase}}\\
    \participants{Sender \underline{$\mathtt{S}$}}{Receiver \underline{$\mathtt{R}$}}   \\
     \mbox{chooses $\vect{r} \R \zo^\ell$, $\vect{e} \in \zo^k$ s.t. $\vect{wt}(\vect{e})=w$} & & \\ 
     \mbox{computes $\vect{c}=\vect{A}(\vect{r} \| \vect{m}) \xor \vect{e}$} & \sends{\vect{c}} \\
    \end{protocol}
\end{figure}

 \begin{figure}
   \begin{protocol}{2}
    \protocolheader{\textbf{Opening phase}}\\
   % \participants{Sender \underline{$\mathtt{S}$}}{Receiver \underline{$\mathtt{R}$}}   \\
     \mbox{computes $ \vect{d} = \left( \vect{m'},\vect{r'} \right) $} & \sends{\vect{d}}  \\
    & & \mbox{computes $\vect{e'}=\vect{c} \xor \vect{A}\left( \vect{r'} \| \vect{m'} \right)$} \\
    & \receives{ {Yes , No}} & \mbox{accepts iff $\vect{wt}(\vect{e'})= w$} \\
   \end{protocol} 
 \end{figure}

  \end{frame}
  
\begin{frame}
\frametitle{Proposed commitment protocol}
 \begin{block}{\alert{Problem}}
  We need a trusted third party for the common matrix $\vect{A}= \left[ \vect{A'} \lVert \vect{A''} \right]$
 \end{block}
\begin{block}{Solution:}
\begin{figure}
    \begin{protocol}{2}
    \protocolheader{\textbf{Setup phase}}\\
    \participants{Sender \underline{$\mathtt{S}$}}{Receiver \underline{$\mathtt{R}$}}   \\
      \mbox{chooses $\vect{A'} \R \Z^{k \times \ell}$} & \sends{\vect{A'}} \\ 
      & \receives{\vect{A''}} & \mbox{chooses $\vect{A''} \R \Z^{k \times v}$ } \\
    \end{protocol}
\end{figure}
The Commitment and Opening phases are the same as in the original scheme 
\end{block}

\end{frame}

\begin{frame}
\frametitle{The commitment protocol: a $\LPN$-based variant}
\begin{block}{$\LPN$ variant}
\begin{itemize}
 \item security directly based on the standard $\LPN$ problem
 \item \textbf{Commit phase:} we set $w'= 2 \cdot \lfloor\tau k\rceil$ and we choose $\vect{e}$ such that $\vect{wt}(\vect{e}) \leq w'$
 \end{itemize}

\end{block}


\begin{block}{Choice of parameters}
According to
\begin{center}
\begin{thebibliography}{XX}
   			\bibitem{LFP:06} 
   			 \textbf{Levieil, \'Eric and Fouque, Pierre-Alain}
  			 \newblock An Improved LPN Algorithm
   				\newblock \emph{Springer Berlin Heidelberg}, 2006
			\end{thebibliography} 
\end{center}

we choose $\ell=768$ and noise rate $\tau = \frac{1}{8}$ $\Rightarrow$ $2^{90}$ bytes of memory to solve $\LPN$
\end{block}

\begin{theorem}
Our commitment scheme is statistically binding and computationally hiding 
\end{theorem}


\end{frame}

\begin{frame}
\frametitle{Proof}
\begin{block}{Statistically binding}
 even if $\mathtt{S}$ is computationally unbounded she cannot cheat with probability greater than $2^{-k}$
\end{block}

\begin{block}{Computationally hiding}
 \begin{itemize}
  \item proof for reduction (single bit message)
  \item we assume that $\mathcal{A}$ is able to break the commitment scheme
  \begin{center}\pgfuseimage{adversary}\end{center}
  \item Let $\mathcal{B}$ an oracle 
  \begin{columns}[c]
    \column{0.50\textwidth}
   \begin{center}
    \pgfuseimage{oracle}
   \end{center}

     \column{0.50\textwidth}
where $\vect{b} = \begin{cases}
		      \mbox{random} & w.p. \: 1/2 \\
		      \vect{A'}\vect{s} \xor \vect{e} & w.p. \: 1/2
                  \end{cases}$
  \end{columns}

  \begin{center} \hspace{1.5cm} 
\end{center}
 \end{itemize}

\end{block}

\end{frame}

\begin{frame}
\frametitle{Proof}

\begin{overprint}
 \onslide<1>\begin{center}\pgfuseimage{0}\end{center}
 \onslide<2>\begin{center}\pgfuseimage{1}\end{center}
 \onslide<3>\begin{center}\pgfuseimage{2}\end{center}
 \onslide<4>\begin{center}\pgfuseimage{3}\end{center} 
 \onslide<5>\begin{center}\pgfuseimage{4}\end{center}
 \onslide<6>\begin{center}\pgfuseimage{5}\end{center}
 \end{overprint}
 
 \begin{overprint}
 \onslide<5>
\begin{block}{case $1)$  }
$\vect{b}$ is random $\Rightarrow$ $\vect{c} = \vect{b} \xor \vect{A''}m$ is a \alert{onetime-pad} encryption\\
\centering $\Rightarrow$ $\mathcal{A}$ guesses w.p. $\frac{1}{2}$\\
\end{block}

\onslide<6>
\begin{block}{case $2)$  }
$\vect{b}$ is a Exact-$\LPN$ sample $\Rightarrow$ $\vect{c}$ is a well formed commitment\\
\centering  $\Rightarrow$ $\mathcal{A}$ guesses w.p. $1$ (by hypothesis)\\ 
\end{block}

\end{overprint}

\end{frame}

\begin{frame}
\frametitle{Proof}

  \begin{center}
  \pgfdeclareimage[height=3.5cm]{3}{images/proof_3}
  \pgfuseimage{3}
  \end{center}

  \begin{block}{case $1)$ and $2)$ }
  Let \mbox{E = {the reduction breaks the Exact-$\LPN$ problem}},  
  \begin{align*}
    \Pr(E) &= \Pr \left( E | \:\vect{b}  = \vect{A'} \vect{s} \xor \vect{e}  \right)  \cdot  \Pr \left( \vect{b}  = \vect{A'} \vect{s} \xor \vect{e}  \right)  + \Pr \left( E   | \: \vect{b} \text{ is random}  \right)   \cdot \Pr \left(   \vect{b} \text{ is random} \right) \\
	  &=  1  \cdot   \frac{1}{2}  +  \frac{1}{2}  \cdot  \frac{1}{2} = \frac{3}{4} \gg 2^{-k}
  \end{align*}
  \begin{center}   
      \alert{Exact-$\LPN$ hardness $\Rightarrow$ Hiding commitment}
  \end{center}
  \end{block}

\end{frame}

 
\begin{frame}
 \frametitle{Conclusions and Open Problems}
 
 \begin{block}{Our contribution}
\begin{itemize}
 \item study the security of our Threshold Public-Key Encryption scheme in the \emph{malicious model}
 \item find statistically hiding commitments
 \item find efficient statistically binding commitments
\end{itemize}
  
 \end{block}

 \begin{block}{$\LPN$ open problems}
  \begin{itemize}
   \item relation between standard $\LPN$ and some variants
   \item $\LPN$ with noise rate $\tau$ imply anything about $\LPN$ with $\tau' < \tau$? Is there a threshold?
   \item how to get some basic primitives from standard $\LPN$?
  \end{itemize}

 \end{block}
 

\end{frame}

\section{Introduzione}
\begin{frame}
 \frametitle{Scenario}
 \begin{center}\pgfuseimage{network}\end{center}
   \begin{block}<1->{Protezione dei contenuti digitali}
    \begin{itemize}
     \item Il commercio elettronico non è ancora percepito come sicuro
     \item Risulta difficile proteggere il diritto d'autore
     \item tecnologie disponibili: protocolli Buyer-Seller
    \end{itemize}
   \end{block}
\end{frame}

\subsection{Protocolli buyer-seller}
\begin{frame}
 \frametitle{I protocolli buyer-seller}
 \begin{itemize}
 \item<1-> I protocolli Buyer-Seller sono schemi interattivi in cui un venditore e un compratore partecipano congiuntamente alla creazione di un contenuto marchiato.
\end{itemize}
\begin{block}<2->{Proprietà:}
 \begin{itemize}
 \item \textbf{Seller:} il contenuto contiene un'informazione che consente di risalire al Buyer
 \item<3-> \textbf{Buyer:} il contenuto non è fruibile al Seller o ad altri utenti 
 \end{itemize}
\end{block} 
 \begin{block}<4->{Soluzioni tecnologiche:}
 \begin{itemize}
 \item crittografia
 \item marchiatura digitale
 \end{itemize}
 \end{block}
\begin{block}<5->{Stato dell'arte:}
\begin{itemize}
\item numerosi protocolli proposti
\item non esistono implementazioni efficienti
\end{itemize}
 \end{block}

\end{frame}

%\texttt{\begin{frame}
% \frametitle{Soluzioni Tecnologiche}
% \begin{block}{La crittografia}
%  \begin{itemize}
%   \item Rende illegibili i dati trasmessi
%   \item Consente una trasmissione sicura
%   \item Non impedisce la diffusione di copie illegali
%  \end{itemize}
% \end{block}
% 
% \begin{block}<2->{La marchiatura elettronica}
%  Marchio digitale (\emph{Watermark}) 
%  \begin{itemize}
%   \item Permette di aggiungere informazioni che attestino la proprietà
%   \item Consente di risalire al compratore fraudolento se il marchio contiene i dati dell'acquirente (\textit{fingerprint})
%  \end{itemize}
% \end{block}
%\end{frame}
%}

\subsection{Scopo della tesi}
\begin{frame}
	\begin{block}<1->{Scopo della tesi}
\begin{itemize}	
 \item<1-> presentare un'implementazione \textit{pratica} di un protocollo buyer-seller
 \item<2->\imp{sicurezza} garantita dai protocolli crittografici
 \item<3-> \imp{efficienza} garantita dalle tecniche di marchiatura.
\end{itemize}	
	\end{block}
	\begin{block}<4->{Soluzione proposta:}
	   \begin{overprint} 
	   \onslide<4->
	   		\begin{thebibliography}{XX}
   			\bibitem{Mina:BSP} 
   			 \textbf{Mina Deng, Tiziano Bianchi, Alessandro Piva, and Bart Preneel}
  			 \newblock Efficient Implementation of a buyer-seller watermarking protocol using a composite signal 									representation
   				\newblock \emph{ACM. MM \& SEC}, 2009
			\end{thebibliography}
	   \end{overprint}
	\end{block}	
\begin{block}<5>{Cosa è stato utilizzato:}
\begin{itemize}
\item Cifratura omomorfica
\item Marchiatura Sicura
\item Firme di gruppo
\item Zero Knowledge Proofs
\end{itemize}
\end{block}
\end{frame}

\subsection{Primitive}
\begin{frame}
\frametitle{Cifratura omomorfica}
\begin{columns}[c]
\begin{column}{0.5\textwidth}
\begin{block}<1->{Cifratura}
\begin{center}
\Large  \onslide<2->{$\textbf{E}($} \onslide<1->{$m$} \onslide<2->$)$ \onslide<3->$=c$
\end{center}
\end{block}
\end{column}
\begin{column}{0.5\textwidth}
\begin{itemize}
\item \onslide<1-> prendere un contenuto $m$
\item \onslide<2-> eseguire un'operazione matematica $\textbf{E}(\:)$
\item \onslide<3-> ottenere $ c \neq m $
\end{itemize}
\end{column}
\end{columns}
\vspace{12pt}
 \begin{block}<4->{Omomorfia additiva}
\begin{itemize}
				\item la somma dei messaggi in chiaro corrisponde al prodotto dei messaggi cifrati:\\
						\hspace{20pt} $E_{pk} (m_1 + m_2) = E_{pk} (m_1) \cdot E_{pk} (m_2)$
				\item il prodotto in chiaro corrisponde all'elevamento a potenza nel cifrato:\\
						\hspace{20pt} $E_{pk} (m_1 \cdot m_2) = [E_{pk} (m_1)]^{m_2}$
\end{itemize}
\end{block} 

\end{frame}

\begin{frame}
\begin{columns}[c]
\begin{column}{0.5\textwidth}
\begin{block}{Marchiatura}
\begin{center}
\Large  \onslide<2->{$\textbf{W}($} \onslide<1->{$I$} \onslide<2->$)$ \onslide<3->$=\textit{I}^{\, \textit{w}}$
\end{center}
\end{block}
\end{column}
\begin{column}{0.5\textwidth}
\begin{itemize}
\item \onslide<1-> prendere un contenuto $I$
\item \onslide<2-> eseguire un'operazione matematica $ \textbf{W}(\:) $
\item \onslide<3-> ottenere $\textit{I}^{\, \textit{w}} \neq \textit{I}$, \alert{ma percettivamente identico}
\end{itemize}
\end{column}
\end{columns}
\vspace{10pt}
\begin{block}<4->{Marchiatura sicura nel dominio cifrato:}
Inserimento di un vettore di bits cifrati in un contenuto digitale
\begin{itemize}
\item<4-> \begin{equation*}
\Large E[I^{w}] = f(I,E[W]).
\end{equation*}
\item<5-> sfrutta proprietà di omomorfia
\item<6-> sfrutta rappresentazione composita per parallelizzare l'operazione di marchiatura 
\end{itemize}
\end{block}
%\begin{block}<4->{Soluzione proposta:}
%\centering \emph{Efficient Composite Embedding}
%\begin{equation*}
%%\begin{split}
%E[b_{C,k}] = E[w_k]^{\sum_{j=0}^{R-1} \Delta_Q(x_{jM+k}, \mathbf{x}) \beta^j}
%%\end{split}
%\label{eq:comp_uC2}
%\end{equation*}
%\end{block}
\end{frame}

\begin{frame}
\frametitle{Firme di gruppo}
Le firme di gruppo sono schemi crittografici che permettono a ciascun membro di produrre firme sotto il nome dello stesso gruppo
   \begin{center}\pgfuseimage{groupsig}\end{center}
\begin{block}<2->{A cosa serve:}
\begin{itemize}
\item mantenere l'\alert<2,2>{anonimato} del compratore
\item garantire la \alert<2,2>{tracciabilità} del compratore
\end{itemize}
%\begin{itemize}
%\item<2-> Anonimity
%\item<3-> Unlinkability
%\item<4-> Unforgeability
%\item<5-> Non-Frameability
%\end{itemize}
\end{block}
\end{frame}

\begin{frame}
\frametitle{I protocolli Zero-Knowledge}
Un protocollo a \imp{conoscenza zero} è un metodo interattivo per provare a qualcun altro che un'affermazione è vera, senza rivelare nient'altro oltre alla veridicità della dichiarazione stessa.
\begin{block}<2->{Attori Coinvolti:}
\begin{itemize}
\item \textbf{Prover} ( Buyer )
\item \textbf{Verifier} ( Seller ) 
\end{itemize}
\end{block}
\begin{block}<3->{A cosa serve:}
\begin{itemize}
\item garantiscono al seller il corretto comportamento del buyer
\item garantiscono i diritti del buyer(Zero-Knowledge)
%\item<3-> Completeness
%\item<4-> Soundness
%\item<5-> Zero-Knowledge
\end{itemize}
\end{block}

%\begin{columns}[c]
%  \column{0.70\textwidth}
%    \begin{itemize}
%     \item <2,4>il \textbf{Prover}  
%     \item <3->il \textbf{Verifier} 
%    \end{itemize}
%  \column{0.25\textwidth}
%		\begin{overprint}
%    \onslide<2>\begin{center}\pgfuseimage{Client}\end{center}
%    \onslide<3>\begin{center}\pgfuseimage{Server}\end{center}
%    \onslide<4>\begin{center}\pgfuseimage{Client}  \\ \pgfuseimage{Server}\end{center} 
%      \end{overprint}
% \end{columns} 
\end{frame}
\section{Presentazione del protocollo}
\begin{frame}
\frametitle{Presentazione dei protocolli}
\begin{overprint}
\onslide<1>
\begin{center}\pgfuseimage{entities}\end{center}
\onslide<2>\begin{center}\pgfuseimage{prot}\end{center}
\onslide<3>\begin{center}\pgfuseimage{prot2}\end{center}
\onslide<4>\begin{center}\pgfuseimage{prot3}\end{center}
\end{overprint}
\begin{overprint}
\onslide<1>
\begin{itemize}
\item Le 4 entità coinvolte nel Protocollo
\end{itemize}
\onslide<2->
\begin{block}<2->{Il protocollo Buyer Seller si suddivide in 3 protocolli:}
\begin{itemize}
\item<2-> \alert<2,2>{Il protocollo di registrazione}
\item<3-> \alert<3,3>{Il protocollo di vendita}
\item<4-> \alert<4,4>{Il protocollo di risoluzione delle dispute}
\end{itemize}
\end{block}
\end{overprint}
\end{frame}

\subsection{Protocollo di registrazione}
\begin{frame}
\frametitle{Il protocollo di registrazione}
\begin{overprint}
    \onslide<1>\begin{center}\pgfuseimage{reg}\end{center}
    \onslide<2>\begin{center}\pgfuseimage{reg2}\end{center}
    \onslide<3>\begin{center}\pgfuseimage{reg3}\end{center} 
    \end{overprint}
\end{frame}

\subsection{Protocollo di vendita}
\begin{frame}
\frametitle{Il protocollo di vendita}
\begin{overprint}
   \onslide<1>\begin{center}\pgfuseimage{sel}\end{center}
    \onslide<2>\begin{center}\pgfuseimage{sel2}\end{center}
    \onslide<3>\begin{center}\pgfuseimage{sel3}\end{center}
    \onslide<4>\begin{center}\pgfuseimage{sel4}\end{center}
    \onslide<5>\begin{center}\pgfuseimage{sel5}\end{center}
    \onslide<6>\begin{center}\pgfuseimage{sel6}\end{center}
\end{overprint}    
\end{frame}

\subsection{Protocollo di risoluzione delle dispute}
\begin{frame}
\frametitle{Il protocollo di risoluzione delle dispute}
\begin{overprint}
   \onslide<1>\begin{center}\pgfuseimage{jud}\end{center}
    \onslide<2>\begin{center}\pgfuseimage{jud2}\end{center}
    \onslide<3>\begin{center}\pgfuseimage{jud3}\end{center}
    \onslide<4>\begin{center}\pgfuseimage{jud4}\end{center}
    \onslide<5>\begin{center}\pgfuseimage{jud5}\end{center}
\end{overprint}    
\end{frame}
\section{Implementazione}
\begin{frame}
\frametitle{Implementazione}
La soluzione proposta è stata implementata in linguaggio C++
 \begin{columns}[l]  
  \column{0.50\textwidth}
\begin{block}<1->{Caratteristiche principali:}
\begin{itemize}
\item<2-> entità modellate come applicazioni indipendenti
\item<3-> scambio messaggi tramite Sockets
\item<4-> Persistenza
\item<5-> \alert<6,6>{Interfaccia Grafica per il Buyer}
\end{itemize}
\end{block}
\begin{block}<6->{Librerie utilizzate:}
\begin{itemize}
\item \textbf{NTL} e \textbf{GMP}
\item \textbf{CIMG:} C++ template image processing toolkit
\item \alert{\textbf{GTK:}} GIMP ToolKit
\end{itemize}
\end{block}
	  \column{0.50\textwidth}
	  \begin{overprint}
	      \onslide<6>\begin{center}\pgfuseimage{interface_b}\end{center}
	      \end{overprint}
   \end{columns}

\end{frame}

\subsection{Diagrammi UML}
\begin{frame}
\frametitle{Diagramma UML Applicazione Seller}
\begin{center}\pgfuseimage{SellerUML}\end{center}
\end{frame}

\begin{frame}
\frametitle{Diagramma UML Applicazione Buyer}
\begin{center}\pgfuseimage{BuyerUML}\end{center}
\end{frame}

\section{Risultati sperimentali}
\subsection{Prestazioni}
\begin{frame}
\frametitle{Protocollo di vendita: Buyer}
\begin{columns}[c]
  \column{0.50\textwidth}
   \begin{center}\pgfuseimage{3}\end{center}
     \column{0.50\textwidth}
   \begin{center}\pgfuseimage{3b}\end{center}
 \end{columns}
 \begin{columns}[c]
  \column{0.50\textwidth}
   \begin{center}Tempo Effettivo\end{center}
     \column{0.50\textwidth}
   \begin{center}Tempo Computazionale\end{center}
 \end{columns}  
\begin{block}{Parametri:}
\begin{itemize}
\item Dimensione Immagine: 512 x 512, 1024 x 1024
\item Sicurezza chiavi: 512, 1024, 2048, 3072 bits
\item Lunghezza marchio: 72, 136 bits
\end{itemize}
\end{block}
\end{frame}

\begin{frame}
\frametitle{Protocollo di vendita: Seller}
\begin{columns}[c]
  \column{0.50\textwidth}
   \begin{center}\pgfuseimage{2}\end{center}
     \column{0.50\textwidth}
   \begin{center}\pgfuseimage{2b}\end{center}
 \end{columns}
 \begin{columns}[c]
  \column{0.50\textwidth}
   \begin{center}Tempo Effettivo\end{center}
     \column{0.50\textwidth}
   \begin{center}Tempo Computazionale\end{center}
 \end{columns}  
\begin{block}{Parametri:}
\begin{itemize}
\item Dimensione Immagine: 512 x 512, 1024 x 1024
\item Sicurezza chiavi: 512, 1024, 2048, 3072 bits
\item Lunghezza marchio: 72, 136 bits
\end{itemize}
\end{block}
\end{frame}

\begin{frame}
\frametitle{Protocollo di vendita: Totale dati scambiati}
   \begin{center}\pgfuseimage{bytes_sell}\end{center}
\begin{block}{Parametri:}
\begin{itemize}
\item Dimensione Immagine: 512 x 512, 1024 x 1024
\item Sicurezza chiavi: 512, 1024, 2048, 3072 bits
\item Lunghezza marchio: 72, 136 bits
\end{itemize}
\end{block}
\end{frame}

\section{Conclusioni}
\begin{frame}
\frametitle{Conclusioni}
\begin{block}{Risultati}
\begin{itemize}
	\item<1-> il protocollo implementato coniuga la sicurezza dei protocolli crittografici con l'efficienza delle tecniche di marchiatura
	\item<2-> il tempo richiesto dal protocollo Buyer-Seller è risultato essere incoraggiante
	
\end{itemize}
\end{block}
\begin{block}<3->{Miglioramenti possibili}
\begin{itemize}
	\item<4-> meccanismo della persistenza aderente al mondo delle Basi di Dati
	\item<5-> protezione dei dati scambiati tramite \emph{SSH}
	\item<6-> funzione del catalogo concepita come ``plug-in''. 
	\end{itemize}
\end{block}
\end{frame}

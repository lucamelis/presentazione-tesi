\section{Introduzione}
\begin{frame}
 \frametitle{Scenario}
   \begin{block}<1->{}
    \begin{itemize}
     \item Il commercio elettronico non è ancora percepito come sicuro
     \item Risulta difficile proteggere il diritto d'autore
     \item tecnologie disponibili: protocolli Buyer-Seller
    \end{itemize}
   \end{block}
\end{frame}

\section{Learning Parity with Noise Problem $\LPN$}

\begin{frame}
\frametitle{Learning Parity with Noise Problem $\LPN$}
\begin{itemize}
 \item Dimension $\ell$ (security parameter), $q \gg \ell$
 \item \textbf{Search}: \underline{find}  $ \vect{s} \in \zol$ given ``noisy random inner products'' \\
 \begin{overprint}
 \onslide<1>
 \begin{align*}
  {\color{blue}{\vect{a_1}}} \R \zol \quad , \quad {\color{red}{b_1}} &= <{\color{blue}{\vect{a_1}}} \: , \: \vect{s}> \: \xor \: e_1 \\
  {\color{blue}{\vect{a_2}}} \R \zol \quad, \quad  {\color{red}{b_2}} &= <{\color{blue}{\vect{a_2}}} \: , \: \vect{s}> \: \xor \: e_2 \\
   & \vdots \\
   {\color{blue}{\vect{a_q}}} \R \zol \quad, \quad  {\color{red}{b_q}} &= <{\color{blue}{\vect{a_q}}} \: , \: \vect{s}> \: \xor \: e_q \\
  \end{align*}

  \onslide<2>
  \vspace*{25pt}
\[
{\color{blue} \vect{A}} = \begin{pmatrix}
            & {\color{blue}\vect{a_1}} & \\
            & \vdots  & \\ 
            & {\color{blue}\vect{a_q}} & 
           \end{pmatrix}  , {\color{red} \vect{b}} = {\color{blue} \vect{A}} \cdot \vect{s} \xor \vect{e} 
\]
 \end{overprint}
  Errors $e_i \gets \chi =$ Bernoulli over $\Z$, param $\tau \in \left( 0,\frac{1}{2} \right] $ 
 \item \textbf{Decision}: \underline{distinguish} $\left( {\color{blue}\vect{a_i}}, {\color{red}\vect{b_i}} \right)$ from {\color{blue}{uniform}} $( {\color{blue}\vect{a_i}}, $ {\color{blue} $b_i$} $)$
 \item decisional and search $\LPN$ are \emph{``polinomially equivalent''} \\ 
 \end{itemize}
\end{frame}

\begin{frame}
 \frametitle{Learning Parity with Noise Problem $\LPN$}

  \begin{block}{$\LPN$ variants}
    \begin{itemize}
      \item Ring $\LPN$
      \item Subspace $\LPN$
      \item Exact $\LPN$
    \end{itemize} 
  \end{block}

  \begin{block}{Hardness of $\LPN$}
  Breaking the search $\LPN$ problem takes time
  \begin{itemize}
   \item $2^{\mathcal{\varTheta}(\mathcal{\ell / \log{\ell}})}$ having the same number of samples $q$
   \item $2^{\mathcal{\varTheta}(\mathcal{\ell / \log{ \log {\ell} } })}$ having $q=poly(\ell)$ samples
   \item $2^{\mathcal{\varTheta}(\mathcal{\ell})}$ having $q= \mathcal{\varTheta}(\ell)$ samples
  \end{itemize}

  \end{block}


\end{frame}

\section{Threshold Public-Key Encryption}

\begin{frame}
\frametitle{Threshold Public-Key Encryption schemes}
\begin{block}{Scenario}
  \begin{itemize}
   \item In public-key cryptography in general, the ability of decrypting or signing is restricted to the owner of the secret key.
   \item $\Rightarrow$ \alert{only one person has all the power} 
  \end{itemize}  
 \end{block}
 
 \begin{block}{Solution}
 \begin{itemize}
  \item Threshold PKE shares trust among a group of users, such that \emph{enough} of them, the \emph{threshold}, is needed to sign or decrypt
  \item The secret key is split into shares and each share is given to a group of users.
 \end{itemize}  
 \end{block}
 
 \begin{block}{Our contribution}
 A Threshold Public-Key Encryption scheme which is:
 \begin{itemize}
   \item based on $\LPN$
   \item secure in the \emph{Semi-honest} model
  \end{itemize}

 \end{block}



\end{frame}

\begin{frame}
\frametitle{Alekhnovich Public Key Encryption scheme}

\begin{block}{\textbf{Key Generation}}
 The sender $\mathtt{S}$ chooses
 \begin{itemize}
  \item a secret key $\vect{s} \R \zol$
  \item $\vect{A} \R \Z^{q \times \ell}$ and the error $\vect{e} \gets \Ber_{\tau}^{q}$ and computes the \emph{pk} as $\left( \vect{A},\vect{b}= \vect{A}\vect{s} \xor \vect{e} \right)$
 \end{itemize}
\end{block}
%   \begin{figure}  
%     \begin{protocol}{2}
%       \protocolheader{\textbf{Key Generation}}\\
%       \participants{Sender \underline{$\mathtt{S}$}}{Receiver \underline{$\mathtt{R}$}}\\
%       &	\hspace*{3cm} & \mbox{Choose a secret key $\vect{s} \R \zol$ }\\
%       & \quad & \mbox{choose a matrix $\vect{A} \R \Z^{q \times \ell}$}\\
%       & \quad & \mbox{and an error vector $\vect{e} \gets \Ber_{\tau}^{q}$} \\
%     \end{protocol}
%  \end{figure}

  \begin{figure}
    \begin{protocol}{2}
      \protocolheader{\textbf{Encryption} of a message bit $m \in \Z$}\\
      \participants{Sender \underline{$\mathtt{S}$}}{Receiver \underline{$\mathtt{R}$}}\\
      \mbox{choose a vector $\vect{f} \gets \Ber_{\tau}^{q}$} & &  \\
      \mbox{compute $\vect{u}= \vect{f} \cdot \vect{A} $ } & & \\
      \mbox{ $c= \left\langle \vect{f},\vect{b} \right\rangle \xor m$} & \sends{( \vect{u}, c ) } \\
    \end{protocol}
 \end{figure}

 \begin{block}{\textbf{Decryption}}
 The receiver $\mathtt{R}$ computes $d = c \xor \left\langle \vect{s},\vect{u} \right\rangle = 
 \dots = \left\langle  \vect{f} , \vect{e} \right\rangle \xor m$ \\ 
\end{block}
\end{frame}

\begin{frame}
\begin{block}{Protocol phases: }
\begin{itemize}
 \item<1-> \alert<1,1>{ \textbf{Key Generation} }
 \item<2-> \alert<2,2>{ \textbf{Key Assembly} }
 \item<3-> \alert<3,3>{ \textbf{Encryption} }
 \item<4-> \alert<4,4>{ \textbf{Partial Decryption} } 
 \item<5-> \alert<5,5>{ \textbf{Finish Decryption} }
 \end{itemize} 
\end{block}
\begin{overprint}
 \onslide<1> \begin{center}\pgfuseimage{key_gen}\end{center}
 \onslide<2> \begin{center}\pgfuseimage{key_assembly}\end{center}
 \onslide<3> \begin{center}\pgfuseimage{enc}\end{center}
 \onslide<4> \begin{center}\pgfuseimage{part_dec}\end{center}
 \onslide<5> \begin{center}\pgfuseimage{finish}\end{center}
\end{overprint}

\end{frame}

\begin{frame}
\begin{block}{Key Generation}
 \begin{itemize}
  \item All the receivers share a matrix $\vect{A}  \R \Z^{q \times \ell}$
  \item Each receiver $\mathtt{R_i}$ \alert{indipendently} choose a secret key $\vect{s_i} \R \zol$ and an error $\vect{e_i} \gets \Ber_{\tau}^{q}$
  \item the public key for $\mathtt{R_i}$ is the pair $\left( \vect{A},\vect{b_i}=\vect{A}\vect{s_i} \xor \vect{e_i} \right)$
 \end{itemize}  
 \end{block}
 
 \begin{block}{Key Assembly}  
 The combined public key is the pair $\left( \vect{A},\vect{b} \right)$, where $\vect{b} = \bigoplus_{i \in I} \vect{b_i}$ ($I$ is the users subset)
 \end{block}

\end{frame}
\begin{frame}

\begin{overprint}
\only<1,2>{
\begin{figure}
\begin{mdframed} 
\begin{protocol}{2}
    \protocolheader{Encryption Phase}\\
    \participants{Sender \underline{$\mathtt{S}$}}{Receivers \underline{$\mathtt{R_i},\mathtt{R_j}$ }} \\
     \left( \vect{C_1},\vect{c_2} \right) \gets \alert{\mathtt{ThLPN.Enc}}(m,\vect{b}) & \phantom{\sends{(\vect{C_1},\vect{c_2})} }\\    
  \end{protocol} 
\end{mdframed} 
\end{figure}
}

\only<3>{
\begin{figure}
\begin{mdframed}
\begin{protocol}{2}
    \protocolheader{Encryption Phase}\\
    \participants{Sender \underline{$\mathtt{S}$}}{Receivers \underline{$\mathtt{R_i},\mathtt{R_j}$ }} \\
     \left( \vect{C_1},\vect{c_2} \right) \gets \alert{\mathtt{ThLPN.Enc}}(m,\vect{b}) &  \sends{(\vect{C_1},\vect{c_2})}   \\    
  \end{protocol}
 
\end{mdframed}
 
\end{figure}
}
 
\end{overprint}


\onslide<2,3>{
\begin{block}{Encryption function (Alekhnovich scheme)}
 \begin{align*}
  \vect{C_1} &= \vect{F} \cdot \vect{A}, \\
  \vect{c_2} &= \vect{F} \cdot \vect{b} \xor \begin{bmatrix} 1 \\ \dots \\ 1 \end{bmatrix} \cdot m 
 \end{align*}
where $\vect{F} := \begin{bmatrix} \vect{f_1} \\ \dots \\ \vect{f_q} \end{bmatrix}$, $\vect{f_i} \gets \Ber_{\tau}^{q}$
 \end{block}
}

\end{frame}

\begin{frame}

\begin{overprint} 
\only<1,2>{
\begin{figure}
  \begin{mdframed} 
    \begin{protocol}{2}
     \protocolheader{Partial Decryption Phase}\\
      \participants{Receiver \underline{$\mathtt{R_i}$}}{Receiver \underline{$\mathtt{R_j}$ }} \\
      \vect{d_i} \gets \alert{\mathtt{ThLPN.Pdec}}(\vect{C_1},\vect{c_2},\vect{s_i})  & \phantom{\sends{\vect{d_i}}}   \\
      & \phantom{\receives{\vect{d_j}}} & \phantom{\vect{d_j} \gets} \\ 
      & & \phantom{\mathtt{ThLPN.Pdec}(\vect{C_1},\vect{c_2},\vect{s_j})} \\
    \end{protocol}
  \end{mdframed}
\end{figure}
}

\only<3>{
\begin{figure}
  \begin{mdframed}
    \begin{protocol}{2}
       \protocolheader{Partial Decryption Phase}\\
      \participants{Receiver \underline{$\mathtt{R_i}$}}{Receiver \underline{$\mathtt{R_j}$ }} \\
      \vect{d_i} \gets \alert{\mathtt{ThLPN.Pdec}}(\vect{C_1},\vect{c_2},\vect{s_i})  & \sends{\vect{d_i}}   \\
      & \phantom{\receives{\vect{d_j}}} & \phantom{\vect{d_j} \gets} \\ 
      & & \phantom{\mathtt{ThLPN.Pdec}(\vect{C_1},\vect{c_2},\vect{s_j})} \\
    \end{protocol}
  \end{mdframed}
\end{figure}
}

\only<4>{
\begin{figure}
  \begin{mdframed}
    \begin{center}
    
    \end{center}
    \begin{protocol}{2}
       \protocolheader{Partial Decryption Phase}\\
      \participants{Receiver \underline{$\mathtt{R_i}$}}{Receiver \underline{$\mathtt{R_j}$ }} \\
      \vect{d_i} \gets \alert{\mathtt{ThLPN.Pdec}}(\vect{C_1},\vect{c_2},\vect{s_i})  & \sends{\vect{d_i}}   \\
      & \phantom{\receives{\vect{d_j}}} & \vect{d_j} \gets \\ 
      & & \alert{\mathtt{ThLPN.Pdec}}(\vect{C_1},\vect{c_2},\vect{s_j}) \\
    \end{protocol}
  \end{mdframed}
\end{figure}
}

\only<5,6>{
\begin{figure}
  \begin{mdframed}
    \begin{protocol}{2}
      \protocolheader{Partial Decryption Phase}\\
      \participants{Receiver \underline{$\mathtt{R_i}$}}{Receiver \underline{$\mathtt{R_j}$ }} \\
      \vect{d_i} \gets \alert{\mathtt{ThLPN.Pdec}}(\vect{C_1},\vect{c_2},\vect{s_i})  & \sends{\vect{d_i}}   \\
      & \receives{\vect{d_j}} & \vect{d_j} \gets \\ 
      & & \alert{\mathtt{ThLPN.Pdec}}(\vect{C_1},\vect{c_2},\vect{s_j}) \\
    \end{protocol}
  \end{mdframed}
\end{figure}
}

% \end{overprint}
% \begin{overprint}
 
\only<2-5>{
\begin{block}{Partial decryption function (Alekhnovich scheme)}
 \begin{equation*}
 \vect{d_i} = \vect{C_1} \cdot \vect{s_i} \xor \vect{\nu_{i}} 
 \end{equation*}
 where  $ \vect{\nu_i} \gets \Ber_{\sigma}^{q} $ 
\end{block}
}
\only<6>{
\begin{block}{Finish decryption}
\begin{itemize}
 \item Each receiver indipendently computes the vector 
\begin{eqnarray*}
\vect{d} =  \vect{c_2} \bigoplus_{i \in I} \left( \vect{d_i} \right)  
	  =  \vect{F}\cdot \vect{e} \xor \begin{bmatrix} 1 \\ \dots \\ 1 \end{bmatrix} \cdot m \bigoplus_{i \in I} \left( \vect{\nu_i} \right) .
\end{eqnarray*}
\item the bit in the vector $\vect{d}$ that is in majority is separately chosen by each receiver as the plaintext $m$
\end{itemize}
\end{block}
}
\end{overprint}


\end{frame}

\begin{frame}
 \frametitle{Protocol Security Analysis }

  \begin{block}{Semi-honest model}
    We make the following two assumptions:
    \begin{enumerate}
      \item The semi-honest party will indeed toss a fair coin
      \item The semi-honest party will send all messages as instructed by the protocol
    \end{enumerate}
  \end{block}

  \begin{block}{Security}
  \begin{itemize}
    \item \textbf{Encryption:} it follows directly from the Alekhnovich's scheme security
    \item \textbf{Decryption:} it follows directly from the $\LPN$ hardness assumption, as each $\mathtt{R}_i$ is generating $\LPN$ samples
  \end{itemize}
  \end{block}
  
\end{frame}
\begin{frame}
  \begin{block}{Relaxed Semi-honest model}
   \begin{itemize}
    \item Semi-honest model not so realistic (\emph{replay attacks} may occur)
    \item \alert{Problem:} if the same message is encrypted multiple times then it is possible to recover information about the secret key from the ciphertexts\\
% 	\item	e.g. let $\vect{d^{i}_1}$ be $3$ encryptions of the same message $m$:
% 		\begin{align*}
% 		\vect{d_1^1} &= \vect{C_1} \cdot \vect{s_1} \xor \vect{\nu_1^1}, \\
% 		\vect{d_1^2} &= \vect{C_1} \cdot \vect{s_1} \xor \vect{\nu_1^2}, \\
% 		\vect{d_1^3} &= \vect{C_1} \cdot \vect{s_1} \xor \vect{\nu_1^3} 
% 		\end{align*}
% 		where $\vect{\nu_1^i},i=1,\dots,3$ are sampled each time according to $\Ber_\tau^q$.\\
% 	\Rightarrow 
   \end{itemize}
  \end{block}

 \begin{block}{Possible solutions}
    \begin{enumerate}
     \item implement the receivers as \emph{stateful} machines (not good in resource-constrained devices)
     \item make use of pseudorandom functions (i.e. deterministic algorithms that simulate truly random functions, given a ``seed'')
    \end{enumerate}
 \end{block}
 
\end{frame}

\section{Commitment Protocol}
\begin{frame}
\frametitle{Commitment Protocols}

\end{frame}

\begin{frame}
 
\end{frame}

\begin{frame}
 \begin{center}\pgfuseimage{adversary}\end{center}

\end{frame}